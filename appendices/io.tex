Cada programa enviado como solución en la OCI es evaluado ejecutándolo con
distintos casos de prueba.
Un caso de prueba corresponde a una instancia del problema y está compuesto por
una entrada y una salida esperada.
La solución será ejecutada una vez por cada caso de prueba, entregándole a esta
la entrada a través de la entrada estándar.
El programa deberá imprimir el resultado del problema por la salida estándar.
La salida del programa es posteriormente comparada con la salida esperada para
el caso de prueba.
El comportamiento por defecto es hacer una comparación \textit{exacta} entre la
salida del programa y la salida esperada para el caso de prueba (similar al
resultado entregado por el comando \verb|diff| en linux);
% TODO referencia a detalles técnicos de entrada y salida
 por lo tanto, es importante que el programa respete el formato de salida esperado
de forma estricta, considerando mayúsculas y minúsculas, espacios y saltos de
línea.
La entrada de cada caso de prueba también seguirá el formato estricto
especificado en el enunciado del problema y por lo tanto las soluciones no
deberán preocuparse de verificarlo.\\

Puede haber problemas donde existe más de una solución posible.
En este caso se instruirá explícitamente en el enunciado que cualquier solución
es válida y en vez de compararse la respuesta de forma exacta con la salida
esperada, esta se evaluará mediante un \emph{verificador}.
Un verificador es un programa que lee la respuesta producida por una solución y
verifica si es correcta.\\

La entrada y salida deben ser siempre por la salida estándar.
A continuación se detallan las formas recomendadas de hacerlo en \java y \cpp.
\paragraph{\cpp.} Existen dos formas de leer y escribir por la entrada y salida
  estándar en \cpp.
  La primera es usando las funciones \code{printf} y \code{scanf} del encabezado
  \code{cstdio}.
  La siguiente forma es a través de \textit{streams} usando lo objetos
  \code{cin} y \code{cout} provistos por el encabezado \code{iostream}.
  Ambas formas están recomendadas y su uso solo depende de preferencias personales.
\paragraph{\java.} La forma recomendada en \java para escribir a la salida estándar es
  utilizando las funciones \code{System.out.println} y \code{System.out.print}.
  Puede ser de utilidad conocer también la función \code{System.out.format}.

  Por otro lado, la lectura de la entrada estándar es más sutil en \java.
  En \java puede resultar tentador utilizar la clase \code{Scanner}; sin embargo, esta forma de lectura puede resultar muy ineficiente pues no
  utiliza un \textit{buffer}.
  Debido a esta ineficiencia, el Comité Científico no garantiza que las
  soluciones que utilicen \code{Scanner} obtengan el mayor puntaje en problemas
  que tengan restricciones ajustadas de tiempo.
  No obstante, su uso queda a criterio del competidor.
  En problemas donde el tamaño de la entrada es pequeño, utilizar \code{Scanner}
  puede no resultar significativo en el tiempo total de ejecución.

  La forma recomendada de leer en \java es utilizando el método \code{readLine}
  de la clase \code{BufferedReader}.
  Para complementar su uso es recomendable utilizar la clase
  \code{StringTokenizer} y funciones como \code{Integer.parseInt}.

%%% Local Variables: 
%%% mode: latex
%%% TeX-master: "../oci-syllabus"
%%% End: 