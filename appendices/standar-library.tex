El Syllabus de la OCI categoriza dentro de tópicos avanzados el uso de algunos
algoritmos y estructuras de datos provistos en la librería estándar de los
lenguajes.
A continuación se detallan, para cada lenguaje, las funciones y estructuras que
cumplen con la expectativa en el Syllabus.

\subsection{Arreglos de tamaño dinámico}
\label{tech-dynamic-array}
\paragraph{\cpp.} La clase \code{vector} implementa arreglos de tamaño dinámico en
  \cpp.
\paragraph{\java.} En \java los arreglos de tamaño dinámico son implementados en la
  clase \code{ArrayList}.

\subsection{Funciones de ordenación}
\label{tech-sort-functions}
\paragraph{\cpp.} Entre otras funciones, el encabezado \code{algorithms} provee la
  función \code{sort}.
  Esta función puede ser utilizada tanto con la clase \code{vector} como con
  arreglos de tamaño fijo.
\paragraph{\java.} Existen dos funciones básicas que implementan ordenación en
  \java.
  La primera es \code{Arrays.sort} que puede ser utilizada con arreglos de
  tamaño fijo.
  La segunda es \code{Collections.sort} que puede ser utilizada con objetos de
  la clase \code{ArrayList}.
  
\subsection{Conjuntos}
\label{tech-set}
\paragraph{\cpp.} Las clases \code{set} y \code{unordered\_set} implementan
  conjuntos.
\paragraph{\java.} Las clases \code{TreeSet} y \code{HashSet} implementan
  conjuntos.

\subsection{Tablas de asociación clave valor}
\label{tech-map}
\paragraph{\cpp.} Las clases \code{map} y \code{unordered\_map} implementan tablas de
  asociación (clave,valor) en \cpp.
\paragraph{\java.} Las clases \code{TreeMap} y \code{HashMap} implementan tablas de
  asociación (clave,valor) en \java.

\subsection{Colas de prioridad}
\label{tech-priority-queue}
\paragraph{\cpp.} La clase \code{priority\_queue} implementa colas de prioridad.
\paragraph{\java.} La clase \code{PriorityQueue} implementa colas de prioridad.

%%% Local Variables: 
%%% mode: latex
%%% TeX-master: "../oci-syllabus"
%%% End: 