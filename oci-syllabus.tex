\documentclass{article}
\usepackage[utf8]{inputenc}
\usepackage[T1]{fontenc}

\newcommand{\basic}{$\heartsuit$}
\newcommand{\advanced}{$\ominus$}
\newcommand{\ultra}{$\otimes$}
\usepackage{theorem}
\usepackage{xspace}
\usepackage{amsmath}
\newtheorem{ejemplo}{Ejemplo}[section]
\newcommand{\java}{Java\xspace}
\newcommand{\cpp}{C\texttt{++}\xspace}
\newcommand{\code}[1]{\texttt{#1}}

\newcommand{\version}{Version 2017}

\pagestyle{myheadings}
\textwidth 126.34 mm
\textheight 189.55 mm
\parindent 20 pt
\parskip 0 pt
\markboth{\sc OCI Syllabus\version}{\sc OCI Syllabus - \version}


\begin{document}
\title{Syllabus para las Olimpiada Chilena de Informática}
\date{}
\maketitle

\section{Versión y estatus de este documento}
Este documento es la versión oficial del Syllabus para la \textbf{OCI 2017}.
El Syllabus es un documento oficial relacionado a la OCI.
Cada año el Comité Científico de la OCI publicará una versión actualizada del Syllabus.

\section{Autores e información de contacto}
Una primera versión de este documento fue publicado para la OCI 2013 a cargo de
Cristian Ruz, Pontificia Universidad Católica de Chile (cruz@ing.puc.cl) y
Jorge Pérez, Universidad de Chile (jperez@dcc.uchile.cl).
La versión actual del documento ha sido modificada a partir de la original por Nicolás
Lehmann actual director del Comité Científico.
Cualquier persona es bienvenida a realizar consultas y sugerencias sobre
Syllabus a la dirección de correo del mantenedor actual (nlehmann@dcc.uchile.cl).

\section{Introducción}\label{sec:intro}
Este documento contiene información acerca de los conceptos mínimos (recomendados)
para la fase regional y nacional de la Olimpiada Chilena de Informática (OCI).
Los objetivos principales de este documento son dos:
\begin{itemize}
  \item Servir de guía para las distintas delegaciones que se encuentren
    entrenando a sus alumnos para la OCI.
  \item Servir como guía al Comité Científico para determinar que problemas son
    adecuadas para la OCI.
\end{itemize}
El Syllabus pretende cumplir estos objetivos dando una categorización de tópicos
en ciencias de la computación.
De forma específica el Syllabus divide los tópicos en tres categorías.
\begin{description}
\item[\basic{} Básico]
  \hspace{1em} \\
  Los tópicos en esta categoría son considerados requisito mínimo.
    Se espera que los participantes los conozcan.
    Estos tópicos podrán aparecer en la descripción de un problema sin una
    explicación adicional.
    El comité científico asegurará que en cada problema al menos una subtarea podrá
    ser resuelta usando solo tópicos dentro de esta categoría (referirse a
    apéndice \ref{problem-description}).
  \item[\advanced{} Avanzado]
    \hspace{1em} \\
    Es recomendable para los participantes conocer estos tópicos
    pues le permitirán obtener el máximo puntaje en cada problema.
  \item[\ultra{} Solo fase nacional]
    \hspace{1em} \\
    Algunos tópicos avanzados serán marcados como solo fase nacional.
    El Comité Científico garantizará que para todos las problemas en la fase
    regional será posible obtener puntaje máximo sin conocimiento sobre los
    tópicos en esta categoría.
\end{description}

Cabe destacar que cada delegación regional es la responsable final de qué
contenidos se entregan a los estudiantes de sus regiones, y puede decidir cubrir
sólo parte de los conceptos descritos en este documento o incluir conceptos
adicionales.

Es intención del Comité Científico aumentar gradualmente la complejidad de los
tópicos cubiertos en este documento.
Por esta razón el Comité Científico puede cada año decidir incluir tópicos
avanzados no cubiertos previamente o considerar como básicos tópicos
considerados anteriormente como avanzados.
No obstante, entendiendo la diversidad en los participantes, es también
compromiso del Comité Científico preparar en cada competencia un conjunto de
tareas adecuado para todos los niveles.
Este compromiso será reflejado en la variedad de subtareas propuestas para cada
problema (referirse a apéndice \ref{problem-description}).

\section{Conceptos fundamentales de programación}

\subsection{Componentes fundamentales de lenguajes de programación}
\begin{itemize}

\item[\basic] Sintaxis y semántica de lenguajes de programación en al menos uno
  de los lenguajes aceptados en la OCI (\java y \cpp): Creación de programas, inclusión de librerías, compilación.

\item[\basic] Variables: declaración de variables, tipos de variables, asignación.

\item[\basic] Instrucciones de I/O: instrucciones de input/output desde la
  entrada/salida estándar término de input, término de línea en output.
  Para detalles técnicos específicos referirse a apéndice \ref{tech-IO}.

\item[\basic] Comparaciones: igualdad, distinto, menor (o igual), mayor (o igual)

\item[\basic] Expresiones Booleanas: uso de conectores lógicos (conjunción, disyunción, negación), uso de paréntesis, evaluación de una expresión.

\item[\basic] Estructuras de control de flujo \texttt{if}/\texttt{else}: control de flujo simple, que requieran uso de expresiones Booleanas compuestas (conjunción, disyunción negación), \texttt{if}/\texttt{else} anidados.

\item[\basic] Estructuras de control de iteración \texttt{for}/\texttt{while}: ciclos \texttt{for}/\texttt{while} simples, con condiciones compuestas (conjunción, disyunción negación) para su terminación, ciclos anidados.

\item[\basic] Funciones: Sintaxis para la declaración de funciones, paso de parámetros, valor de retorno, llamado a funciones, uso
de funciones en expresiones, llamado a funciones de librerías incluidas.
\end{itemize}


%%% Local Variables: 
%%% mode: latex
%%% TeX-master: "../oci-syllabus"
%%% End: 

\subsection{Tipos de datos básicos y operaciones}
\begin{itemize}
\item[\basic] Tipos Booleanos: uso de tipos Booleanos en expresiones lógicas (conjunción, disyunción, negación), en control de flujo y estructuras iterativas.

\item[\basic] Enteros: manejo de números enteros, lectura/escritura de enteros desde/hacia salida/entrada estándar.

\item[\basic] Expresiones aritméticas sobre enteros: uso de operadores entre enteros incluido resto/módulo (\texttt{\%}),
división entre enteros, exponenciación y otras funciones de librerías matemáticas básicas, 
expresiones bien formadas en uso de paréntesis.

\item[\basic] Strings: lectura y escritura de strings, manipulación básica de strings, . 

\end{itemize}


%%% Local Variables: 
%%% mode: latex
%%% TeX-master: "../oci-syllabus"
%%% End: 

\subsection{Estructuras de datos simples}

\begin{itemize}
\item[\basic{}] Arreglos: arreglos de tipos de datos básicos, creación de un
  arreglo de tamaño fijo, creación de un arreglo de tamaño especificado por una
  variable, acceso a las posiciones del arreglo.

\item[\basic{}] Recorrer arreglos: recorrer arreglos para calcular alguna operación entre
  sus elementos, para obtener máximos, mínimos, promedios, contar la aparición
  de valores, etc., comparar arreglos, determinar elementos en común,
  subsecuencias, etc.

\item[\basic{}] Matrices (arreglos de dos dimensiones): matrices de tipos
  básicos, creación de matriz de tamaño variable, acceso a las posiciones de la
  matriz.

\item[\basic{}] Recorrer matrices: recorrer matrices para calcular alguna operación entre
  sus elementos, para obtener máximos, mínimos, promedios, contar la aparición
  de valores, etc., comparar matrices, determinar la existencia de submatrices,
  etc.

\item[\basic{}] Arreglos de tamaño dinámico: creación y manipulación de arreglos
  de tamaño dinámico provistos por la librería del lenguaje de programación escogido.
  Acceso a una posición arbitraria en el arreglo.
  Aumentar o disminuir tamaño de un arreglo añadiendo o eliminando elementos del
  final de este.
  Para información técnica detallada referirse a apéndice \ref{tech-dynamic-array}.

\item[\advanced{}] Definición de nuevos tipos de datos simples:
  definición de estructuras para encapsular un conjunto de datos simples, por
  ejemplo, estructura para representar pares de números.
  Definición de operaciones (funciones) sobre nuevos tipos de datos.

\item[\advanced{}] Arreglos de tipos distintos de los básicos, por ejemplo,
  arreglo de pares de números.
\end{itemize}

%%% Local Variables: 
%%% mode: latex
%%% TeX-master: "../oci-syllabus"
%%% End: 

\subsection{Recursión}
\begin{itemize}
\item[\basic] Concepto de recursión 
\item[\basic] Funciones matemáticas recursivas
\item[\basic] Implementación de funciones recursivas básicas
\item[\basic] Implementación de funciones mutuamente recursivas
\end{itemize}

%%% Local Variables: 
%%% mode: latex
%%% TeX-master: "../oci-syllabus"
%%% End: 

\section{Algoritmos y estructuras de datos}
\subsection{Ordenación y búsqueda}
\begin{itemize}
\item[\basic] Búsqueda lineal en secuencias de datos (no necesariamente ordenados).
\item[\basic] Búsqueda lineal en secuencias de datos ordenados.
\item[\advanced]
  Algoritmos de ordenación: algún algoritmo simple de ordenación de valores
  enteros como InsertSort, SelectSort, BubbleSort; de menor a mayor, de mayor a
  menor y con criterio alternativo.
\item[\ultra] Búsqueda binaria para encontrar elementos en una secuencia ordenada
  y en su forma general para encontrar puntos en funciones monótonas
  de dominio discreto.
\end{itemize}


%%% Local Variables: 
%%% mode: latex
%%% TeX-master: "../oci-syllabus"
%%% End: 

\subsection{Estructuras de datos y algoritmos provistos por la librería estándar}
\begin{itemize}
\item[\advanced]
Funciones de ordenación: uso de las funciones de la librería estándar para
ordenar arreglos de tipos básicos.
Uso de las funciones de ordenación para arreglos de tipos distintos de básicos
y con criterio de comparación personalizado.
Para información técnica detallada referirse a apéndice \ref{tech-sort-functions}.
\item[\advanced]
Uso de estructuras que implementen conjuntos de elementos con consciencia de su
eficiencia por sobre una implementación basada en arreglos y búsqueda lineal.
No se espera conocimiento sobre la implementación subyacente de estas
estructuras (tablas de hash, árboles auto-balanceantes).
Para información técnica detallada referirse a apéndice \ref{tech-set}.
\item[\advanced]
Uso de estructuras que implementen tablas de asociación clave valor con
consciencia de su eficiencia por sobre una implementación basada en arreglos y
búsqueda lineal.
No se espera conocimiento sobre la implementación subyacente de estas
estructuras (tablas de hash, árboles auto-balanceantes).
Para información técnica detallada referirse a apéndice \ref{tech-map}.
\item[\advanced]
Uso de estructuras que implementen colas de prioridad con consciencia de su
eficiencia por sobre una implementación basada en arreglos y búsqueda lineal.
No se espera conocimiento sobre la implementación subyacente de estas
estructuras (árboles auto-balanceantes).
Para información técnica detallada referirse a apéndice \ref{tech-priority-queue}.
\end{itemize}


%%% Local Variables: 
%%% mode: latex
%%% TeX-master: "../oci-syllabus"
%%% End: 

\subsection{Algorítmos numéricos}
\begin{itemize}
\item[\basic] Dígitos de un número entero: descomponer un número entero en sus
  dígitos constituyentes.
\item[\basic] Operaciones usando módulo: aritmética modular, reportar los
  últimos dígitos de un número grande.
\item[\basic] Números primos: determinación de primalidad de un número por
  búsqueda exhaustiva.
\item[\advanced] Números primos: determinación de primalidad de un número por
  búsqueda exhaustiva hasta la raíz cuadrada.
{\new\item[\advanced] Algoritmo de Euclides para  calcular el máximo común divisor (MCD) de dos números enteros.}
\end{itemize}

%%% Local Variables:
%%% mode: latex
%%% TeX-master: "../oci-syllabus"
%%% End:


\subsection{Grafos}
\begin{itemize}
\item[\ultra] Noción de grafos como matriz de adyacencia y como listas de
  adyacencia: grafos simples (a lo más una arista entre cada par de nodos)
  dirigidos y no dirigidos, con y sin loops.
\item[\ultra] Noción de grafos con peso: grafos con peso en nodos, peso en aristas.
\item[\ultra] Noción de grafo implícito en estructura de adyacencia (ej.
  grafo generado en un tablero donde los vecinos son los casilleros de arriba,
  abajo, izquierda y derecha).
\item[\ultra] Algoritmos de recorrido en grafos: bfs y dfs.
\item[\ultra] Algoritmos de distancia mínima entre nodos en grafos con peso
  (dijkstra) y sin peso (bfs).
\end{itemize}


%%% Local Variables: 
%%% mode: latex
%%% TeX-master: "../oci-syllabus"
%%% End: 

\newpage
\appendix
\section{Descripción de un problema de la OCI}
\label{problem-description}
Esta sección describe la estructura típica de un problema en la OCI, la división en
distintas subtareas, la forma de asignar puntaje, cómo se corrigen durante la
competencia y el feedback que reciben los participantes.

\subsection{Estructura de los problemas}

En general, todo problema de la OCI comienza con una descripción (entregando
contexto y motivando típicamente desde una situación cercana a la realidad).
Después de analizar el problema, el estudiante podrá entender cómo resolverla
usando técnicas de programación. El siguiente ejemplo muestra un enunciado
resumido del problema ``Palabras'' de la OCI 2013.

\begin{ejemplo}
Para crear palabras, una tribu extraterrestre usa solo un conjunto reducido de
letras y todas las combinaciones de letras forman palabras válidas.
Se requiere saber cuántas palabras de un largo fijo se pueden formar para un
conjunto de letras dado.
Por ejemplo, si las letras que se usan son `\$' y `\&', entonces las palabras
distintas de largo 3 que se pueden formar son \$\$\$, \$\$\&, \$\&\$, \$\&\&,
\&\$\$, \&\$\&, \&\&\$ y \&\&\&, que en total son 8.
No se pide el número total, si no que basta con conocer los últimos 4 dígitos de
ese valor.
\end{ejemplo}

Luego de la descripción del problema se describen la forma en que se entregará la
entrada/input junto con las restricciones de los valores, y la forma en que debe
ser entregada la salida/output por parte de la solución.
Por ejemplo, en el caso del problema ``Palabras'' descrita anteriormente, la
entrada era un par de números enteros $N$ y $L$ donde $N$ corresponde a la
cantidad de letras usadas y $L$ al largo de las palabras.
La salida debía ser un entero correspondiente a los $4$ últimos dígitos de la
cantidad de palabras de largo $L$ que se pueden crear con $N$ letras.
En el caso de la entrada la restricción era que $N$ podía tomar valores entre
$1$ y $10$, y que $L$ podía tomar valores entre $1$ y $10^9-1$.

Posteriormente, el enunciado del problema describe las distintas subtareas que se
someterán a evaluación y el puntaje por completar cada una.
Cada subtarea tendrá restricciones adicionales al problema que pueden hacer que su
solución sea más simple.
En general las distintas subtareas van aumentando su nivel de dificultad y una
solución a la última subtarea (la más difícil) debiera abarcar todas las subtareas
anteriores.
Esta es una característica común, pero no una regla estricta.
Es posible que la solución a una subtarea no resuelva las subtareas anteriores.
El siguiente ejemplo muestra las subtareas del problema ``Palabras'' de la OCI 2013.

\begin{ejemplo}
Se probarán varios casos con los siguientes puntajes y restricciones: 
\begin{center} 
\begin{tabular}{|r|c|c|c} \hline
&{\bf Puntaje} & {\bf Restricción} \\ \hline
Subtarea 1 &30 puntos & $1\leq L<4$ \\
Subtarea 2 &30 puntos & $4\leq L< 12$ \\
Subtarea 3 &20 puntos & $12\leq L<10^8$ \\
Subtarea 4 &20 puntos & $10^8\leq L<10^9$ \\ \hline
\end{tabular}
\end{center}
\end{ejemplo}\bigskip

El puntaje total de un problema, sumando los puntajes de cada subtarea que lo componen
será siempre de 100 puntos, tal como en el ejemplo anterior.

% En la mayoría de los casos el problema finalizará indicando la cantidad de
% tiempo que se destinará
% para la ejecución de cada caso de prueba, y un ejemplo de entrada/input y salida/output.
En el caso del ejemplo anterior se puede notar cómo el problema va aumentando la dificultad
para cada una de las subtareas.
A continuación se hace un análisis del problema ``Palabras'' y la solución más
simple para cada una de las posibles tareas.

Lo primero que se puede observar en el problema ``Palabras'' es que la cantidad
de palabras distintas de largo $L$ que se pueden hacer con $N$ letras es
simplemente $N^L$.
La segunda observación es que ese número podría ser muy grande dependiendo de
los valores de $N$ y $L$ y que por lo tanto podría ser muy lento de calcular y 
su valor podría no ser almacenable en un tipo de dato básico.
Con esas observaciones se puede entregar las siguientes estrategias:

\begin{description}
\item[Subtarea 1]: Dado que $N$ es siempre menor o igual
a $10$ y $L$ es menor que $4$, sabemos que $N^L$ tendrá menos de 4 dígitos.
Luego, en este caso bastaba con entregar como salida el valor $N^L$.
Más aún el cálculo de $N^L$ se podría hacer de forma muy simple: 
dado que los valores posibles de $L$ son $1,2,3$, 
se podría hacer una implementación en donde la salida fuera, 
o bien $N$, o $N\times N$, o $N\times N\times N$, dependiendo del valor de $L$,
lo que puede ser implementado incluso usando solo \texttt{if}/\texttt{else} y 
operaciones básicas entre enteros.

\item[Subtarea 2]: En este caso el valor $N^L$ podría tener más de 4 dígitos por 
lo que sería incorrecto simplemente entregar $N^L$ como salida.
Además, los valores de $L$ podrían ser muchos (entre $4$ y $11$)
por lo que era necesario hacer una iteración para calcular $N^L$.
Dado que el valor máximo para $N^L$ era $10^{11}$ este número podría ser
almacenado en una variable y posteriormente del resultado obtener los últimos 4 dígitos.
Obtener los dígitos se podía hacer dividiendo y restando, o usando el 
operador \texttt{\%}.

\item[Subtarea 3]: En este caso el valor de $N^L$ podría ser gigante (hasta $10^{10^8}$)
por lo que es imposible calcular primero el valor ya que no es almacenable en ningún tipo
de dato.
La observación que se podía hacer es que como sólo se necesitaban los últimos $4$ dígitos
entonces podríamos usar aritmética modular de manera tal de multiplicar $N$ consigo mismo
$L$ veces usando una iteración, pero tomando el módulo (\texttt{\%}) $10000$ luego de cada multiplicación. De esta
forma todos los números intermedios tendrían a lo mas $8$ dígitos, y cuando se terminara
el cálculo se tendría exactamente el valor pedido.


\item[Subtarea 4]: Esta es la tarea más difícil.
  En este caso el valor de $L$ puede ser tan grande
($10^9-1$) que incluso hacer una iteración desde $1$ hasta $L$ podría necesitar
demasiado tiempo (mayor a 1 segundo que era el tiempo máximo de ejecución).
Una solución a este problema que obtenía todo el puntaje era notar que o bien
$L$ o $L-1$ se podía dividir por 2 para hacer el computo más rápido de la
siguiente forma: (1) si $L$ es par, entonces se calcular $N^{L/2}$ y luego se
eleva al cuadrado, (2) si $L$ es impar, entonces se calcula $N^{(L-1)/2}$ y
luego se eleva al cuadrado y se multiplica por $N$.
En ambos casos se debía tomar módulo $10000$ después de cada cálculo intermedio.
Esta solución llevaba a la mitad el tiempo necesario para calcular $N^L$.
Otra solución general, que alcanzaba la mejor eficiencia era usar
\emph{exponenciación rápida}%
\footnote{Este tópico no está incluido dentro de los tópicos para OCI 2013-2014}
que es esencialmente la idea anterior pero dividiendo $L$ de manera recursiva.
Esencialmente se calcula una función recursiva $\operatorname{expRap}(N,L)$ de manera tal que
\begin{eqnarray*}
\operatorname{expRap}(N,L) & = &\operatorname{expRap}(N,{L}/{2}) \times 
\operatorname{expRap}(N,{L}/{2}) \text{\;\;\;\; si $L$ es par} \\
\operatorname{expRap}(N,L) & = & N\times \operatorname{expRap}(N,{(L-1)}) \text{\;\;\;\; si $L$ es impar}
\end{eqnarray*}
con caso base $\operatorname{expRap}(N,L)=N$ y de manera que en cada llamada intermedia 
se tome el módulo $10000$.

\end{description}
Este problema es un problema claramente difícil para obtener el puntaje completo.
Note además que cualquier solución para una de las tareas resolvía satisfactoriamente todas
las tareas anteriores.

\subsection{Evaluación y feedback}

Cuando un competidor envía una posible solución para ser evaluada durante la competencia,
el sistema de evaluación ejecutará su programa con todos los casos de prueba disponibles 
y una vez obtenidos los resultados se le notificará al competidor del puntaje obtenido.
Los casos de prueba están divididos en subtareas y para obtener el puntaje en
una subtarea se debe tener correctos todos los casos de prueba de la subtarea.
El sistema de evaluación entregará al participante el detalle de todos los casos
que fueron resueltos correctamente y aquellos en los que hubo un error, pero el
contenido de los casos de prueba siempre permanecerá oculto.
El puntaje final de un participante corresponde a la suma del puntaje máximo que
haya sacado en cada problema.

%%% Local Variables: 
%%% mode: latex
%%% TeX-master: "../oci-syllabus"
%%% End: 

\section{Operaciones de entrada y salida (IO)}
\label{tech-IO}
Cada programa enviado como solución en la OCI es evaluado ejecutándolo con
distintos casos de prueba.
Un caso de prueba corresponde a una instancia del problema y está compuesto por
una entrada y una salida esperada.
La solución será ejecutada una vez por cada caso de prueba, entregándole a esta
la entrada a través de la entrada estándar.
El programa deberá imprimir el resultado del problema por la salida estándar.
La salida del programa es posteriormente comparada con la salida esperada para
el caso de prueba.
El comportamiento por defecto es hacer una comparación \textit{exacta} entre la
salida del programa y la salida esperada para el caso de prueba (similar al
resultado entregado por el comando \verb|diff| en linux);
% TODO referencia a detalles técnicos de entrada y salida
 por lo tanto, es importante que el programa respete el formato de salida esperado
de forma estricta, considerando mayúsculas y minúsculas, espacios y saltos de
línea.
La entrada de cada caso de prueba también seguirá el formato estricto
especificado en el enunciado del problema y por lo tanto las soluciones no
deberán preocuparse de verificarlo.\\

Puede haber problemas donde existe más de una solución posible.
En este caso se instruirá explícitamente en el enunciado que cualquier solución
es válida y en vez de compararse la respuesta de forma exacta con la salida
esperada, esta se evaluará mediante un \emph{verificador}.
Un verificador es un programa que lee la respuesta producida por una solución y
verifica si es correcta.\\

La entrada y salida deben ser siempre por la salida estándar.
A continuación se detallan las formas recomendadas de hacerlo en \java y \cpp.
\paragraph{\cpp.} Existen dos formas de leer y escribir por la entrada y salida
  estándar en \cpp.
  La primera es usando las funciones \code{printf} y \code{scanf} del encabezado
  \code{cstdio}.
  La siguiente forma es a través de \textit{streams} usando lo objetos
  \code{cin} y \code{cout} provistos por el encabezado \code{iostream}.
  Ambas formas están recomendadas y su uso solo depende de preferencias personales.
\paragraph{\java.} La forma recomendada en \java para escribir a la salida estándar es
  utilizando las funciones \code{System.out.println} y \code{System.out.print}.
  Puede ser de utilidad conocer también la función \code{System.out.format}.

  Por otro lado, la lectura de la entrada estándar es más sutil en \java.
  En \java puede resultar tentador utilizar la clase \code{Scanner}; sin embargo, esta forma de lectura puede resultar muy ineficiente pues no
  utiliza un \textit{buffer}.
  Debido a esta ineficiencia, el Comité Científico no garantiza que las
  soluciones que utilicen \code{Scanner} obtengan el mayor puntaje en problemas
  que tengan restricciones ajustadas de tiempo.
  No obstante, su uso queda a criterio del competidor.
  En problemas donde el tamaño de la entrada es pequeño, utilizar \code{Scanner}
  puede no resultar significativo en el tiempo total de ejecución.

  La forma recomendada de leer en \java es utilizando el método \code{readLine}
  de la clase \code{BufferedReader}.
  Para complementar su uso es recomendable utilizar la clase
  \code{StringTokenizer} y funciones como \code{Integer.parseInt}.

%%% Local Variables: 
%%% mode: latex
%%% TeX-master: "../oci-syllabus"
%%% End: 

\section{Algoritmos y estructuras de la librería estándar}
El Syllabus de la OCI categoriza dentro de tópicos avanzados el uso de algunos
algoritmos y estructuras de datos provistos en la librería estándar de los
lenguajes.
A continuación se detalla para cada lenguaje las funciones y estructuras que
cumplen con la expectativa en el Syllabus.

\subsection{Arreglos de tamaño dinámico}
\label{tech-dynamic-array}
\paragraph{\cpp.} La clase \code{vector} implementa arreglos de tamaño dinámico en
  \cpp.
\paragraph{\java.} En \java los arreglos de tamaño dinámico son implementados en la
  clase \code{ArrayList}.

\subsection{Funciones de ordenación}
\label{tech-sort-functions}
\paragraph{\cpp.} Entre otras funciones, el encabezado \code{algorithms} provee la
  función \code{sort}.
  Esta función puede ser utilizada tanto con la clase \code{vector} como con
  arreglos de tamaño fijo.
\paragraph{\java.} Existen dos funciones básicas que implementan ordenación en
  \java.
  La primera es \code{Arrays.sort} que puede ser utilizada con arreglos de
  tamaño fijo.
  La segunda es \code{Collections.sort} que puede ser utilizada con objetos de
  la clase \code{ArrayList}.
  
\subsection{Conjuntos}
\label{tech-set}
\paragraph{\cpp.} La clases \code{set} y \code{unordered\_set} implementan
  conjuntos.
\paragraph{\java.} Las clases \code{TreeSet} y \code{HashSet} implementan
  conjuntos.

\subsection{Tablas de asociación clave valor}
\label{tech-map}
\paragraph{\cpp.} Las clases \code{map} y \code{unordered\_map} implementan tablas de
  asociación clave valor en \cpp.
\paragraph{\java.} Las clases \code{TreeMap} y \code{HashMap} implementan tablas de
  asociación clave valor en \java.

\subsection{Colas de prioridad}
\label{tech-priority-queue}
\paragraph{\cpp.} La clase \code{priority\_queue} implementa colas de prioridad.
\paragraph{\java.} La case \code{PriorityQueue} implementa colas de prioridad.

%%% Local Variables: 
%%% mode: latex
%%% TeX-master: "../oci-syllabus"
%%% End: 

\end{document}