
\begin{itemize}
\item[\basic{}] Arreglos: arreglos de tipos de datos básicos, creación de un
  arreglo de tamaño fijo, creación de un arreglo de tamaño especificado por una
  variable, acceso a las posiciones del arreglo.

\item[\basic{}] Recorrer arreglos: recorrer arreglos para calcular alguna operación entre
  sus elementos, para obtener máximos, mínimos, promedios, contar la aparición
  de valores, etc., comparar arreglos, determinar elementos en común,
  subsecuencias, etc.

\item[\basic{}] Matrices (arreglos de dos dimensiones): matrices de tipos
  básicos, creación de matriz de tamaño variable, acceso a las posiciones de la
  matriz.

\item[\basic{}] Recorrer matrices: recorrer matrices para calcular alguna operación entre
  sus elementos, para obtener máximos, mínimos, promedios, contar la aparición
  de valores, etc., comparar matrices, determinar la existencia de submatrices,
  etc.

\item[\basic{}] Arreglos de tamaño dinámico: creación y manipulación de arreglos
  de tamaño dinámico provistos en la librería del lenguaje de programación escogido.
  Acceso a una posición arbitraria en el arreglo.
  Aumentar o disminuir tamaño de un arreglo añadiendo o eliminando elementos del
  final de este.
  Para información técnica detallada referirse a apéndice \ref{tech-dynamic-array}.

\item[\advanced{}] Definición de nuevos tipos de datos simples:
  definición de estructuras para encapsular un conjunto de datos simples, por
  ejemplo, estructura para representar pares de números.
  Definición de operaciones (funciones) sobre nuevos tipos de datos.

\item[\advanced{}] Arreglos de tipos distintos de los básicos, por ejemplo,
  arreglo de pares de números.
\end{itemize}

%%% Local Variables: 
%%% mode: latex
%%% TeX-master: "../oci-syllabus"
%%% End: 